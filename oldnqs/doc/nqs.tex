\documentclass[12pt]{article}
\pdfoutput=1
\usepackage{bm}% bold math
\usepackage{graphicx}
\graphicspath{}
%\usepackage{tabularx}
\usepackage{amsfonts}
\usepackage{amsmath}
\usepackage{amssymb}
\usepackage{amsbsy}
\usepackage{bbm}
\usepackage{hyperref}
\usepackage{mathdots}
\usepackage{booktabs}
\setlength{\heavyrulewidth}{1.5pt}
\setlength{\abovetopsep}{4pt}
\usepackage[margin=0.7in]{geometry}
\usepackage{nicefrac}
\usepackage{subcaption}
\newenvironment{psmallmatrix}
  {\left[\begin{matrix}}
  {\end{matrix}\right]}
  \usepackage{listings}


\begin{document}

\noindent Notation:
\begin{itemize}
\item visible nodes $\vec{x}  \in \mathbb{R}^M$ - gaussian units (position coordinates)
\item hidden nodes $\vec{h} \in \mathbb{R}^N$ - binary units
\item visible biases $\vec{a} \in \mathbb{R}^M$
\item hidden biases $\vec{b} \in \mathbb{R}^N$
\item interaction weights $\hat{W} \in \mathbb{R}^{M\times N}$
\item $j$th column vector of weight matrix $\vec{W}_j \in \mathbb{R}^M$
\item $i$th row vector of weight matrix $\vec{W}_i^T \in \mathbb{R}^N$ 
\item variational parameters $\vec{\theta} = ( a_0, \cdots, a_{M-1},b_0,\cdots,b_{N-1}, W_{0,0},\cdots,W_{M-1,N-1} )$ \\
\end{itemize}

\noindent "Energy" of a configuration of nodes:
\begin{align*}
E(\vec{x},\vec{h}) &= \frac{1}{2\sigma^2} \sum_{i=0}^{M-1}(x_i - a_i)^2 - \sum_{j=0}^{N-1} b_j h_j -
\frac{1}{\sigma^2} \sum_{i=0}^{M-1}\sum_{j=0}^{N-1} x_i W_{ij} h_j\\
&= \frac{1}{2\sigma^2} \| \vec{x} - \vec{a} \|^2 - \vec{b}^T \vec{h} - \frac{1}{\sigma^2}\vec{x}^T \hat{W} \vec{h}
\end{align*}

\noindent Use the marginal probability to represent the wavefunction as a function of the RBM inputs:
\begin{align*}
\Psi_{NQS}(\vec{x}) &= \frac{1}{Z} \sum_{\vec{h}} e^{-E(\vec{x},\vec{h})}\\
&= \frac{1}{Z} \sum_{\vec{h}} \exp 
\left(
-\frac{1}{2\sigma^2} \| \vec{x} - \vec{a} \|^2 + \vec{b}^T \vec{h} + \frac{1}{\sigma^2}\vec{x}^T \hat{W} \vec{h}
 \right)\\
 &= \frac{1}{Z} \exp 
\left(
-\frac{1}{2\sigma^2} \| \vec{x} - \vec{a} \|^2
\right) \sum_{ \{ h_j \} } \exp \left( \sum_{j=0}^{N-1} b_j h_j + \frac{1}{\sigma^2} \sum_{i=0}^{M-1}\sum_{j=0}^{N-1} x_i W_{ij} h_j  \right)\\
&= \frac{1}{Z} \exp 
\left(
-\frac{1}{2\sigma^2} \| \vec{x} - \vec{a} \|^2
\right)
\prod_{j=0}^{N-1} \sum_{ h_j=0 }^1 \exp \left( b_j h_j + \frac{1}{\sigma^2} \sum_{i=0}^{M-1} x_i W_{ij} h_j  \right)\\
&= \frac{1}{Z} \exp \left( -\frac{1}{2\sigma^2} \| \vec{x}-\vec{a} \|^2 \right) \prod_{j=0}^{N-1} \left( 1 + \exp \left( b_j + \frac{1}{\sigma^2} \vec{x}^T \vec{W}_j \right) \right)
\end{align*}


\noindent Hamiltonian for $P$ $D$-dimensional particles in a harmonic oscillator $(\hbar = m = 1)$ is
\begin{align*}
\hat{H} = \frac{1}{2} \sum_{p = 0}^{P-1} \left( -\nabla_p^2 + \sum_{d=0}^{D-1} \omega_d^2 r_{p,d}^2 \right) + \sum_{p<q} V_{int} (r_{pq}), \ \ r_{pq} = \sqrt{ \sum_{d=0}^{D-1} (r_{p,d}-r_{q,d})^2 },
\end{align*}
\noindent where $\nabla_p^2$ is the Laplacian with respect to the coordinates of the $p$th particle
\begin{align*}
\nabla_p &= \sum_{d=0}^{D-1} \hat{n}_d \frac{\partial}{\partial r_{p,d}},\\
\nabla_p^2 &= \nabla_p \cdot \nabla_p = \sum_{d=0}^{D-1} \frac{\partial^2}{\partial r_{p,d}^2},\\
\end{align*}
\noindent and $\hat{n_d}$, $d = 0, ..., D-1$, denote the elementary unit vectors in each of the $D$ dimensions.\\


\noindent We need to vectorize the coordinates of all the particles to input into the RBM:
\begin{align*}
\vec{x} = ( r_{0,0}, \cdots, r_{0,D-1}, \cdots, r_{P-1,0}, \cdots, r_{P-1,D-1} ) \in \mathbb{R}^M, \ M = PD
\end{align*}
\noindent The visible biases and the rows of the weight matrix are organized in the same way. The mapping between coordinates and the visible nodes is:
\begin{align*}
r_{p,d} &= x_{Dp+d}\\
x_i &= r_{\text{floor}(i/D), i \text{mod} D}
\end{align*}

\noindent Now we can write our representation of the wavefunction as a function of all coordinates:
\begin{align*}
\Psi_{NQS}(\vec{R}) &=\frac{1}{Z} \exp \left( -\frac{1}{2\sigma^2} \sum_{p=0}^{P-1}\sum_{d=0}^{D-1} (r_{p,d} - a_{Dp+d})^2 \right) \prod_{j=0}^{N-1} \left( 1 + \exp \left( b_j + \frac{1}{ \sigma^2} \sum_{p=0}^{P-1}\sum_{d=0}^{D-1} r_{p,d} W_{Dp+d,j} \right) \right)\\
&=\frac{1}{Z} \exp (A(\vec{R})) \prod_{j=0}^{N-1} \exp\left( \ln \left(  1 + \exp \left( b_j + \frac{1}{ \sigma^2} \sum_{p=0}^{P-1}\sum_{d=0}^{D-1} r_{p,d} W_{Dp+d,j} \right) \right) \right) \\
&=\frac{1}{Z} \exp (A(\vec{R})) \exp \left( \sum_{j=0}^{N-1} \ln \left(  1 + \exp \left( b_j + \frac{1}{ \sigma^2} \sum_{p=0}^{P-1}\sum_{d=0}^{D-1} r_{p,d} W_{Dp+d,j} \right) \right) \right) \\
&= \frac{1}{Z} \exp  \Big( A(\vec{R})+ B(\vec{R}) \Big) \\ \\
A(\vec{R}) &\equiv -\frac{1}{2\sigma^2} \sum_{p=0}^{P-1}\sum_{d=0}^{D-1} (r_{p,d} - a_{Dp+d})^2 = - \frac{1}{2\sigma^2}\sum_{i=0}^{M-1} (x_i-a_i)^2 = -\frac{1}{2\sigma^2} \| \vec{x} - \vec{a} \|^2\\
B(\vec{R}) &\equiv \sum_{j=0}^{N-1} \ln \left(  1 + \exp \left( B_j (\vec{R}) \right) \right)\\
B_j(\vec{R}) &\equiv b_j + \frac{1}{ \sigma^2}\sum_{p=0}^{P-1}\sum_{d=0}^{D-1} r_{p,d} W_{Dp+d,j} = b_j +  \frac{1}{ \sigma^2} \sum_{i=0}^{M-1} x_i W_{i,j} = b_j + \frac{1}{ \sigma^2} \vec{x}\cdot \vec{W}_j
\end{align*}

\noindent If the particles are low-density bosons with a hard-core interaction:
\begin{align*}
V_{int} (r_{pq}) = 
\begin{cases} 
	\infty, & r_{pq} \leq a,\\
	0, & r_{pq} > a,
   \end{cases}
\end{align*}
and if the particles are electrons ($ e = 1$, $S=0$) with the Coulomb interaction:
\begin{align*}
V_{int} (r_{pq}) = \frac{1}{r_{pq}}
\end{align*}

\noindent The behaviors due to these interactions are accounted for in the trial wavefunction by a corresponding Jastrow factor so that the full trial wavefunction is 
\begin{align*}
\Psi _T (\vec{R}) = \Psi_{NQS}(\vec{R}) \exp(J(\vec{R})) = \frac{1}{Z} \exp \Big( A(\vec{R})+B(\vec{R})+J(\vec{R}) \Big)
\end{align*}
For bosons:
\begin{align*}
J_B(\vec{R}) = \sum_{p<q} \ln \left( 1-\frac{a}{r_{pq}} \right)
\end{align*}
For electrons:
\begin{align*}
J_C(\vec{R}) = \sum_{p<q} r_{pq}
\end{align*}

\noindent The local energy is defined as
\begin{align*}
E_L = \frac{1}{\Psi_T} \hat{H} \Psi_T
\end{align*}

\noindent We want an analytical expression for the local energy in terms of our variational parameters $\vec{\theta}$. First, we have that 

\begin{align*}
 \frac{1}{ \Psi_T }\nabla_p \Psi_T  &=  \nabla_p  A  + \nabla_p  B  +\nabla_p  J \\ \\ 
 \frac{1}{ \Psi_T }\nabla_p^2  \Psi_T  &= \nabla_p^2  A  + \nabla_p^2  B  +\nabla_p^2  J   + \Big( \nabla_p A  + \nabla_p B  +\nabla_p  J  \Big)^2\\ \\
\end{align*}

\noindent Derivatives of $A$:

\begin{align*}
A(\vec{R}) &\equiv -\frac{1}{2\sigma^2} \sum_{p=0}^{P-1}\sum_{d=0}^{D-1} (r_{p,d} - a_{Dp+d})^2,\\
%%%%%%%%%%%%%%%%%%%%%%%%%%%%%%%%%%%%%%%%%%%%%%%%%%%
\nabla_p A &= -\frac{1}{2\sigma^2}
\sum_{d=0}^{D-1} \hat{n_{d}}  \frac{\partial}{\partial r_{k,d}} \bigg\{  \sum_{p'=0}^{P-1}\sum_{d'=0}^{D-1} (r_{p',d'} - a_{Dp'+d'})^2 \bigg\} \\
&= -\frac{1}{2\sigma^2} \sum_{d=0}^{D-1} \sum_{p'=0}^{P-1}\sum_{d'=0}^{D-1}  2 (r_{p',d'} - a_{Dp'+d'}) \delta_{p',p} \delta_{d',d}\hat{n_{d}}= \frac{1}{\sigma^2} \sum_{d=0}^{D-1}(a_{Dp+d}-r_{p,d}) \hat{n_{d}}\\
%%%%%%%%%%%%%%%%%%%%%%%%%%%%%%%%%%%%%%%%%%%%%%%%%%%
\nabla_p^2 A &=\frac{1}{\sigma^2} \sum_{d=0}^{D-1} \hat{n}_d \frac{\partial}{\partial r_{p,d}} \cdot \sum_{d'=0}^{D-1}(a_{Dp+d'}-r_{p,d'}) \hat{n_{d'}}=\frac{1}{\sigma^2} \sum_{d=0}^{D-1} \sum_{d'=0}^{D-1}(-1) \delta_{d',d} = - \frac{D}{\sigma^2}
\end{align*}

\noindent Derivatives of $B_j$:
\begin{align*}
B_j(\vec{R}) &\equiv b_j + \frac{1}{\sigma^2} \sum_{p=0}^{P-1}\sum_{d=0}^{D-1} r_{p,d} W_{Dp+d,j}\\
\nabla_p B_j &= \sum_{d=0}^{D-1} \hat{n}_{d} \frac{\partial}{\partial r_{p,d}} \bigg\{ b_j + \frac{1}{\sigma^2} \sum_{p'=0}^{P-1}\sum_{d'=0}^{D-1} r_{p',d'} W_{Dp'+d',j} \bigg\}\\
 &= \frac{1}{\sigma^2} \sum_{d=0}^{D-1} \sum_{p'=0}^{P-1}\sum_{d'=0}^{D-1} W_{Dp'+d',j} \delta_{p',p} \delta_{d',d} \hat{n}_d = \frac{1}{\sigma^2} \sum_{d=0}^{D-1} W_{Dp+d,j} \hat{n}_d \\
 \nabla_p^2 B_j &= 0\\
\end{align*}

\noindent Derivatives of $B$:
\begin{align*}
B(\vec{R}) &\equiv \sum_{j=0}^{N-1} \ln \Big(  1 + \exp \big( B_j (\vec{R}) \big) \Big)\\
\nabla_p  B  &= \sum_{j=0}^{N-1}\nabla_p \bigg\{  \ln \Big(  1 + \exp \big( B_j \big) \Big) \bigg\} = \sum_{j=0}^{N-1} \frac{1}{1+\exp(B_j)} \nabla_p  \bigg\{  1 + \exp \big( B_j \big) \bigg\} \\
&=\sum_{j=0}^{N-1} \frac{\exp(B_j)}{1+\exp(B_j)} \nabla_p B_j =  \frac{1}{\sigma^2} \sum_{j=0}^{N-1} \frac{\exp(B_j)}{1+\exp(B_j)}\sum_{d=0}^{D-1} W_{Dp+d,j} \hat{n}_d\\
&= \frac{1}{\sigma^2} \sum_{j=0}^{N-1} \sum_{d=0}^{D-1} \frac{W_{Dp+d,j}}{\exp(-B_j)+1}  \hat{n}_d\\
\nabla_p^2 B &= \frac{1}{\sigma^2} \nabla_p \cdot \sum_{j=0}^{N-1} \sum_{d=0}^{D-1} \frac{W_{Dp+d,j}}{\exp(-B_j)+1}  \hat{n}_d = \frac{1}{\sigma^2} \sum_{j=0}^{N-1} \sum_{d=0}^{D-1} W_{Dp+d,j} \frac{\exp(-B_j)}{(\exp(-B_j)+1)^2} \nabla_p B_j \cdot \hat{n}_d \\
&= \frac{1}{\sigma^4}\sum_{j=0}^{N-1} \sum_{d=0}^{D-1} W_{Dp+d,j}^2 \frac{\exp(-B_j)}{(\exp(-B_j)+1)^2}\\
\end{align*}

\noindent Derivative of the distance between particles $p$ and $q$, $r_{pq}$:

\begin{align*}
r_{pq} &= \left( \sum_{d=0}^{D-1} (r_{p,d}-r_{q,d})^2 \right)^{1/2}\\
\nabla_k r_{pq} &= \frac{1}{2} \left( \sum_{d=0}^{D-1} (r_{p,d}-r_{q,d})^2 \right)^{-1/2} \nabla_k \bigg\{ \sum_{d=0}^{D-1} (r_{p,d}-r_{q,d})^2 \bigg\} \\
&= \frac{1}{2 r_{pq}} \sum_{d=0}^{D-1} 2(r_{p,d}-r_{q,d})(\delta_{k,p}-\delta_{k,q}) \hat{n}_{d} = \frac{\delta_{k,p}-\delta_{k,q}}{r_{pq}} \sum_{d=0}^{D-1} (r_{p,d}-r_{q,d}) \hat{n}_{d} \\ \\
\end{align*}

\noindent For bosons:
\begin{align*}
J_B(\vec{R}) &= \sum_{p<q} \ln \left( 1-\frac{a}{r_{pq}} \right)\\
\nabla_k  J_B  &= \sum_{p<q} \frac{1}{1-\frac{a}{r_{pq}}} \nabla_{k} \left( 1-\frac{a}{r_{pq}} \right) 
= \sum_{p<q} \frac{r_{pq}}{r_{pq}-a} \frac{a}{r_{pq}^2} \nabla_k r_{pq} \\
&= \sum_{p<q} \frac{a}{r_{pq}(r_{pq}-a)} \frac{1}{r_{pq}} \sum_{d=0}^{D-1} (r_{p,d}-r_{q,d}) (\delta_{k,p}-\delta_{k,q}) \hat{n}_{d} \\
&=\frac{1}{2} \sum_{d=0}^{D-1} \sum_{p\neq q} \frac{a (r_{p,d}-r_{q,d})}{r_{pq}^2(r_{pq}-a)} (\delta_{k,p}-\delta_{k,q}) \hat{n}_{d} \\
&=\frac{1}{2} \sum_{d=0}^{D-1} \left[  \sum_{p\neq q} \frac{a (r_{p,d}-r_{q,d})}{r_{pq}^2(r_{pq}-a)} \delta_{k,p} - \sum_{p\neq q} \frac{a (r_{p,d}-r_{q,d})}{r_{pq}^2(r_{pq}-a)} \delta_{k,q} \right] \hat{n}_{d} \\
&=\frac{1}{2} \sum_{d=0}^{D-1} \left[  \sum_{q\neq k} \frac{a (r_{k,d}-r_{q,d})}{r_{kq}^2(r_{kq}-a)}  - \sum_{p\neq k} \frac{a (r_{p,d}-r_{k,d})}{r_{pk}^2(r_{pk}-a)} \right] \hat{n}_{d} \\
&=\frac{1}{2} \sum_{d=0}^{D-1} \left[ 2  \sum_{p\neq k} \frac{a (r_{k,d}-r_{p,d})}{r_{kp}^2(r_{kp}-a)} \right] \hat{n}_{d} =  a \sum_{d=0}^{D-1}  \sum_{p=0, \ p\neq k}^{P-1} \frac{r_{k,d}-r_{p,d}}{r_{kp}^2(r_{kp}-a)}  \hat{n}_{d} \\
%%%%%%%%%%%%%%%%%%%%%%%%%%%%%%%%%%%
\nabla_k^2 J_B &= a \sum_{d=0}^{D-1} \hat{n}_d \frac{\partial}{\partial r_{k,d}} \cdot \sum_{d'=0}^{D-1}  \sum_{p=0, \ p\neq k}^{P-1} \frac{r_{k,d'}-r_{p,d'}}{r_{kp}^2(r_{kp}-a)}  \hat{n}_{d'} \\
&= a \sum_{d=0}^{D-1} \sum_{d'=0}^{D-1}  \sum_{p=0, \ p\neq k}^{P-1} \frac{\partial}{\partial r_{k,d}} \left\{  \frac{r_{k,d'}-r_{p,d'}}{r_{kp}^2(r_{kp}-a)} \right\} \delta_{d',d} \\
&= a \sum_{d=0}^{D-1} \sum_{p=0, \ p\neq k}^{P-1} \frac{\partial}{\partial r_{k,d}} \left\{  \frac{r_{k,d}-r_{p,d}}{r_{kp}^2(r_{kp}-a)} \right\} \\
&= a \sum_{d=0}^{D-1} \sum_{p=0, \ p\neq k}^{P-1} \left[ 
(r_{k,d}-r_{p,d}) \frac{\partial}{\partial r_{k,d}} \left\{ \frac{1}{r_{kp}^2(r_{kp}-a)} \right\} + \frac{\partial}{\partial r_{k,d}} \big\{ r_{k,d}-r_{p,d} \big\} \frac{1}{r_{kp}^2(r_{kp}-a)}
\right] \\
&= a \sum_{d=0}^{D-1} \sum_{p=0, \ p\neq k}^{P-1} \Bigg[ 
(r_{k,d}-r_{p,d}) 
\bigg[ 
\frac{1}{r_{kp}^2} \frac{\partial}{\partial r_{k,d}}  \left\{ \frac{1}{r_{kp}-a} \right\}
+\frac{\partial}{\partial r_{k,d}} \bigg\{ \frac{1}{r_{kp}^2}  \bigg\} \frac{1}{r_{kp}-a}
\bigg]
 + \frac{1}{r_{kp}^2(r_{kp}-a)}
\Bigg] \\
%%%%%%%%%%%%%%%%%%%%%%%%%%%%%%%%%%%
&= a \sum_{d=0}^{D-1} \sum_{p=0, \ p\neq k}^{P-1} \Bigg[ 
(r_{k,d}-r_{p,d}) \frac{\partial r_{kp}}{\partial r_{k,d}} 
\bigg[ 
\frac{-1}{r_{kp}^2(r_{kp}-a)^2}
+ \frac{-2}{r_{kp}^3} 
 \frac{1}{r_{kp}-a}\bigg]
 + \frac{1}{r_{kp}^2(r_{kp}-a)}
\Bigg] \\
%%%%%%%%%%%%%%%%%%%%%%%%%%%%%%%%%%%
&= a \sum_{d=0}^{D-1} \sum_{p=0, \ p\neq k}^{P-1}
\frac{1}{r_{kp}^2(r_{kp}-a)}
 \Bigg[ 
\frac{(r_{k,d}-r_{p,d})^2}{r_{kp}} 
\bigg[ 
\frac{-1}{r_{kp}-a}
+ \frac{-2}{r_{kp}} \bigg]
 + 1 \Bigg] \\
 %%%%%%%%%%%%%%%%%%%%%%%%%%%%%%%%%%%
&= a \sum_{d=0}^{D-1} \sum_{p=0, \ p\neq k}^{P-1}
\frac{1}{r_{kp}^2(r_{kp}-a)}
 \Bigg[ 
\frac{(r_{k,d}-r_{p,d})^2}{r_{kp}} 
\bigg[ 
\frac{2a-3r_{kp}}{r_{kp}(r_{kp}-a)} \bigg]
 + 1 \Bigg] \\
  %%%%%%%%%%%%%%%%%%%%%%%%%%%%%%%%%%%
&= a \sum_{d=0}^{D-1} \sum_{p=0, \ p\neq k}^{P-1}
\frac{1}{r_{kp}^2(r_{kp}-a)}
 \Bigg[ 
 \frac{(r_{k,d}-r_{p,d})^2 (2a-3r_{kp})}{r_{kp}^2(r_{kp}-a)}  + 1
 \Bigg] \\
\end{align*}
\begin{align*}
\nabla_k^2 J_B &= a  \sum_{p=0, \ p\neq k}^{P-1}
\frac{1}{r_{kp}^2(r_{kp}-a)}
\sum_{d=0}^{D-1} \Bigg[ 
 \frac{(r_{k,d}-r_{p,d})^2 (2a-3r_{kp})}{r_{kp}^2(r_{kp}-a)}  + 1
 \Bigg] \\
 %%%%%%%%%%%%%%%%%
 &= a  \sum_{p=0, \ p\neq k}^{P-1}
\frac{1}{r_{kp}^2(r_{kp}-a)} \Bigg[ 
 \frac{(2a-3r_{kp})}{r_{kp}^2(r_{kp}-a)}\sum_{d=0}^{D-1} (r_{k,d}-r_{p,d})^2 + D
 \Bigg] \\
 %%%%%%%%%%%%%%%%%
  &= a  \sum_{p=0, \ p\neq k}^{P-1}
\frac{1}{r_{kp}^2(r_{kp}-a)} \Bigg[ 
 \frac{(2a-3r_{kp})}{(r_{kp}-a)} + D
 \Bigg] \\
  %%%%%%%%%%%%%%%%%
  &= a  \sum_{p=0, \ p\neq k}^{P-1}
\frac{1}{r_{kp}^2(r_{kp}-a)} \Bigg[ 
\frac{(2-D)a+(D-3)r_{kp}}{(r_{kp}-a)}
 \Bigg] \\
   %%%%%%%%%%%%%%%%%
  &= a  \sum_{p=0, \ p\neq k}^{P-1}
\frac{(2-D)a+(D-3)r_{kp}}{r_{kp}^2(r_{kp}-a)^2} \\
\end{align*}

\noindent For charged particles:
\begin{align*}
J_C (\vec{R}) &= \sum_{p<q} r_{pq} \\
\nabla_k J_C &= \sum_{p<q} \nabla_k r_{pq} = \sum_{p<q} \frac{\delta_{k,p}-\delta_{k,q}}{r_{pq}} \sum_{d=0}^{D-1} (r_{p,d}-r_{q,d}) \hat{n}_{d}\\
&= \frac{1}{2} \sum_{p\neq q} \frac{\delta_{k,p}}{r_{pq}} \sum_{d=0}^{D-1} (r_{p,d}-r_{q,d}) \hat{n}_{d} - \frac{1}{2} \sum_{p\neq q} \frac{\delta_{k,q}}{r_{pq}} \sum_{d=0}^{D-1} (r_{p,d}-r_{q,d}) \hat{n}_{d}\\
&= \frac{1}{2} \sum_{q\neq k} \frac{1}{r_{kq}} \sum_{d=0}^{D-1} (r_{k,d}-r_{q,d}) \hat{n}_{d} + \frac{1}{2} \sum_{p\neq k} \frac{1}{r_{pk}} \sum_{d=0}^{D-1} (r_{k,d}-r_{p,d}) \hat{n}_{d}= \sum_{d=0}^{D-1} \sum_{p\neq k} \frac{r_{k,d}-r_{p,d}}{r_{kp}} \hat{n}_{d}\\
%%%%%%%%%%%%%%%%%%%%%%%%%%%
\nabla_k^2 J_C 
&= \sum_{d=0}^{D-1} \sum_{p\neq k} \left[
(r_{k,d}-r_{p,d}) \frac{\partial}{\partial r_{k,d}} \left\{ \frac{1}{r_{kp}} \right\} + \frac{\partial}{\partial r_{k,d}} \Big\{ r_{k,d}-r_{p,d} \Big\} \frac{1}{r_{kp}} 
 \right]\\
 &= \sum_{d=0}^{D-1} \sum_{p\neq k} \left[
-\frac{(r_{k,d}-r_{p,d})}{r_{kp}^2} \frac{\partial r_{kp}}{\partial r_{k,d}}  + \frac{1}{r_{kp}} 
 \right]
 = \sum_{d=0}^{D-1} \sum_{p\neq k} \left[
-\frac{(r_{k,d}-r_{p,d})}{r_{kp}^2} \frac{(r_{k,d}-r_{p,d})}{r_{kp}}  + \frac{1}{r_{kp}} 
 \right]\\
 &= \sum_{d=0}^{D-1} \sum_{p\neq k} \frac{1}{r_{kp}}
  \left[ 1
-\frac{(r_{k,d}-r_{p,d})^2}{r_{kp}^2}  
 \right] = \sum_{p \neq k} \frac{1}{r_{kp}} 
 \left[  D - \frac{1}{r_{kp}^2} \sum_{d=0}^{D-1} (r_{k,d}-r_{p,d})^2
 \right] = (D-1) \sum_{p\neq k} \frac{1}{r_{kp}} \\
\end{align*}

\noindent Depending on what type of particles we have, substitute $J$ with $J_B$ or $J_C$. Putting all this together...
\begin{align*}
E_L &= \frac{1}{\Psi} \left[ \frac{1}{2} \sum_{p=0}^{P-1} -\nabla_p^2 + \frac{1}{2}\sum_{d=0}^{D-1} \omega_d^2r_{p,d}^2 \right] \Psi = -\frac{1}{2} \sum_{p=0}^{P-1} \frac{1}{\Psi} \nabla_p^2 \Psi + \frac{1}{2} \sum_{p=0}^{P-1} \sum_{d=0}^{D-1} \omega_d^2r_{p,d}^2\\
&= -\frac{1}{2} \sum_{p=0}^{P-1} 
\bigg[ \nabla_p^2  A   + \nabla_p^2  B   + \nabla_p^2  J  + \Big( \nabla_p  A   + \nabla_p  B   + \nabla_p  J \Big)^2 \bigg]
+ \frac{1}{2} \sum_{p=0}^{P-1} \sum_{d=0}^{D-1} \omega_d^2r_{p,d}^2\\
%%%%%%%%%%%%%%%%%%%%%%%%%%%%%%%%
&= -\frac{1}{2} \sum_{p=0}^{P-1} 
\Bigg[ -\frac{D}{\sigma^2}  
+ \frac{1}{\sigma^4} \sum_{j=0}^{N-1} \sum_{d=0}^{D-1} W_{Dp+d,j}^2 \frac{\exp(-B_j)}{(\exp(-B_j)+1)^2}
 + \nabla_p^2  J  \\
& \hspace*{20mm}+ \bigg( \frac{1}{\sigma^2} \sum_{d=0}^{D-1}(a_{Dp+d}-r_{p,d}) \hat{n_{d}}  + \frac{1}{\sigma^2} \sum_{j=0}^{N-1} \sum_{d=0}^{D-1} \frac{W_{Dp+d,j}}{\exp(-B_j)+1}  \hat{n}_d
 + \nabla_p  J \bigg)^2 \Bigg]
+ \frac{1}{2} \sum_{p=0}^{P-1} \sum_{d=0}^{D-1} \omega_d^2r_{p,d}^2\\
%%%%%%%%%%%%%%%%%%%%%%%%%%%%%%%%%%%%%%%%%%
&= \frac{PD}{2\sigma^2} - \frac{1}{2\sigma^4} \sum_{p=0}^{P-1} \sum_{d=0}^{D-1} \sum_{j=0}^{N-1} W_{Dp+d,j}^2 \frac{\exp(-B_j)}{(\exp(-B_j)+1)^2} - \frac{1}{2} \sum_{p=0}^{P-1} \nabla_p^2  J\\
& \hspace*{5 mm} -\frac{1}{2} \sum_{p=0}^{P-1} \bigg( \frac{1}{\sigma^2} \sum_{d=0}^{D-1}(a_{Dp+d}-r_{p,d}) \hat{n_{d}}  + \frac{1}{\sigma^2} \sum_{d=0}^{D-1} \sum_{j=0}^{N-1} \frac{W_{Dp+d,j}}{\exp(-B_j)+1}  \hat{n}_d
 + \nabla_p  J \bigg)^2 + \frac{1}{2} \sum_{p=0}^{P-1} \sum_{d=0}^{D-1} \omega_d^2r_{p,d}^2\\
\end{align*}

\noindent Let's write the local energy in terms of our RBM inputs $\vec{x}$, i.e. switch from sums over particles and dimensions to sums over inputs. Here, I'll define some vectors that will make computation of the local energy simpler and faster. Store the quantities $B_j$ in a vector called $\vec{B} \in \mathbb{R}^N$ (not to be confused with the function $B(\vec{R})$), and store the quantities
\begin{align*}
f_{j} = \frac{1}{\exp(-B_j)+1}
\end{align*}
in a vector $\vec{f} \in \mathbb{R}^N$. In the code, $\vec{f}$ is called \texttt{sigmoidB}. Also, store the squares of the frequencies and positions in vectors $\vec{\Omega^2}, \vec{x^2} \in \mathbb{R}^M$,
\begin{align*}
(\Omega^2)_i &= (\omega_{i \text{mod} D})^2\\
(x^2)_i &= x_i^2
\end{align*}
so that the last double sum in the local energy becomes a simple dot product.
 Then the local energy becomes:

\begin{align*}
E_L &= \frac{M}{2\sigma^2} - \frac{1}{2} \sum_{j=0}^{N-1} \exp(-B_j) \bigg\| \frac{f_j \vec{W}_j}{\sigma^2} \bigg\|^2 - \frac{1}{2} \sum_{p=0}^{P-1} \nabla_p^2  J\\
& \hspace*{5 mm} -\frac{1}{2} \sum_{p=0}^{P-1} \bigg( \frac{1}{\sigma^2} \sum_{d=0}^{D-1} \Big( a_{Dp+d}-x_{Dp+d} + \vec{W}_{Dp+d}^T \vec{f} \  \Big) \hat{n}_d
 + \nabla_p  J \bigg)^2 + \frac{1}{2} \vec{\Omega^2} \cdot \vec{x^2}\\
&= \frac{1}{2} \Bigg[ \frac{M}{\sigma^2} - \sum_{j=0}^{N-1} \exp(-B_j) \bigg\| \frac{f_j \vec{W}_j}{\sigma^2} \bigg\|^2 + \vec{\Omega^2} \cdot \vec{x^2} \Bigg]\\
& \hspace*{5 mm} - \frac{1}{2} \sum_{p=0}^{P-1} \Bigg[  \nabla_p^2  J +\bigg( \frac{1}{\sigma^2} \sum_{d=0}^{D-1} \Big( a_{Dp+d}-x_{Dp+d} + \vec{W}_{Dp+d}^T \vec{f} \  \Big) \hat{n}_d
 + \nabla_p  J \bigg)^2 \Bigg] \\
\end{align*}

\noindent The quantum force on the $p$th particle is defined as 
\begin{align*}
\vec{F}_p(\vec{R}) = 2 \frac{1}{\Psi} \nabla_p \Psi.
\end{align*}

\noindent Using the derivatives we have already calculated, we have

\begin{align*}
\vec{F}_p(\vec{R}) &= 2 \Big( \nabla_p A + \nabla_p B +\nabla_p J \Big)\\
&= 2\bigg( \frac{1}{\sigma^2} \sum_{d=0}^{D-1} \Big( a_{Dp+d}-x_{Dp+d} + \vec{W}_{Dp+d}^T \vec{f} \  \Big) \hat{n}_d
 + \nabla_p  J \bigg)\\
\end{align*}

\noindent Notice that the quantum force on the $p$th particle $\vec{F}_p$ simplifies our expression for the local energy:

\begin{align*}
E_L &= \frac{1}{2} \Bigg[ \frac{M}{\sigma^2} - \sum_{j=0}^{N-1} \exp(-B_j) \bigg\| \frac{f_j \vec{W}_j}{\sigma^2} \bigg\|^2 + \vec{\Omega^2} \cdot \vec{x^2} \Bigg]\\
& \hspace*{5 mm} - \frac{1}{2} \sum_{p=0}^{P-1} \Bigg[  \nabla_p^2  J +\bigg( \frac{1}{\sigma^2} \sum_{d=0}^{D-1} \Big( a_{Dp+d}-x_{Dp+d} + \vec{W}_{Dp+d}^T \vec{f} \  \Big) \hat{n}_d
 + \nabla_p  J \bigg)^2 \Bigg] \\
 &= \frac{1}{2} \Bigg[ \frac{M}{\sigma^2} - \sum_{j=0}^{N-1} \exp(-B_j) \bigg\| \frac{f_j \vec{W}_j}{\sigma^2} \bigg\|^2 + \vec{\Omega^2} \cdot \vec{x^2} - \sum_{p=0}^{P-1} \bigg(  \nabla_p^2  J + \bigg\|\frac{1}{2} \vec{F}_p \bigg\|^2 \bigg) \Bigg] \\
\end{align*}


\noindent For stochastic gradient descent, we will also need the gradient of the local energy with respect to the RBM parameters $\vec{\theta}=(\vec{a},\vec{b},\hat{W})$:
\begin{align*}
\frac{\partial \langle E_L \rangle}{\partial \theta_k} = 2 \left( \left\langle E_L \frac{1}{\Psi_T} \frac{\partial \Psi_T}{\partial \theta_k} \right\rangle - \left\langle E_L \right\rangle \left\langle \frac{1}{\Psi_T} \frac{\partial \Psi_T}{\partial \theta_k} \right\rangle \right)= 2 \left( \left\langle E_L \frac{\partial \ln \Psi_T}{\partial \theta_k} \right\rangle - \left\langle E_L \right\rangle \left\langle \frac{\partial \ln \Psi_T}{\partial \theta_k} \right\rangle \right)\\
\end{align*}



\noindent The derivatives of our new wavefunction with respect to the variational parameters are given by:
\begin{align*}
\frac{\partial \ln  \Psi_T}{\partial \vec{a}} &= \frac{\partial A(\vec{a})}{\partial \vec{a}} = \frac{\vec{x}-\vec{a}}{\sigma^2}\\
\frac{\partial \ln \Psi_T}{\partial \vec{b}} &=  \frac{\partial B(\vec{b},\hat{W})}{\partial \vec{b}} =  \vec{f} \\
\frac{\partial \ln \Psi_T}{\partial \hat{W}} &=  \frac{\partial B(\vec{b},\hat{W})}{\partial \hat{W}}
= \frac{\vec{x} \vec{f}^T}{\sigma^2}
\end{align*}






\noindent The Langevin and Fokker-Planck equations give a new position $y$ from the old position $x$:
\begin{align}
y = x + d\Delta tF(x) + \xi \sqrt{\Delta t},
\end{align}
where $d=0.5$ is the diffusion constant and $\Delta t \in [0.001,0.01]$ is a chosen time step. The transition probability is given by the Green's function
\begin{align}
G(y,x)=\frac{1}{(4\pi d \Delta t)^{3N/2}} \exp \left( -\frac{(y-x-d\Delta t F(x))^2}{4 d \Delta t} \right),
\end{align}
so that the Metropolis-Hastings acceptance ratio is
\begin{align}
A(y,x) = \min \{ 1, P(y,x) \},
\end{align}
where 
\begin{align*}
P(y,x) &= \frac{G(x,y) | \Psi_T(y) | ^2}{G(y,x) | \Psi_T(x) | ^2}\\
&= \exp \left( -\frac{(x-y-d\Delta t F(y))^2}{4 d \Delta t} \right) \exp \left( \frac{(y-x-d\Delta t F(x))^2}{4 d \Delta t} \right) \frac{| \Psi_T(y) | ^2}{ | \Psi_T(x) | ^2}\\
&= \exp \left( -\frac{(x-y)^2-2(x-y)d\Delta t F(y) + d^2 \Delta t^2 F(y)^2}{4d\Delta t} \right)\\
& \ \ \  \times \exp \left( \frac{(y-x)^2-2(y-x)d\Delta t F(x) + d^2 \Delta t^2 F(x)^2}{4d\Delta t} \right) \frac{| \Psi_T(y) | ^2}{ | \Psi_T(x) | ^2}\\
&= \exp \left( \frac{2(x-y)d\Delta t (F(y)+F(x)) + d^2 \Delta t^2 (F(x)^2-F(y)^2)}{4d\Delta t} \right) \frac{| \Psi_T(y) | ^2}{ | \Psi_T(x) | ^2}\\
&= \exp \left( \frac{2(x-y) (F(y)+F(x)) + d \Delta t (F(x)^2-F(y)^2)}{4} \right) \frac{| \Psi_T(y) | ^2}{ | \Psi_T(x) | ^2}\\
&= \exp \left( \frac{1}{2}(x-y) (F(y)+F(x)) + \frac{1}{4} d \Delta t (F(x)^2-F(y)^2) \right) \frac{| \Psi_T(y) | ^2}{ | \Psi_T(x) | ^2}\\
\end{align*}




\end{document}