\documentclass[12pt]{article}
\pdfoutput=1
\usepackage{bm}% bold math
\usepackage{graphicx}
\graphicspath{}
%\usepackage{tabularx}
\usepackage{amsfonts}
\usepackage{amsmath}
\usepackage{amssymb}
\usepackage{amsbsy}
\usepackage{bbm}
\usepackage{hyperref}
\usepackage{mathdots}
\usepackage{booktabs}
\setlength{\heavyrulewidth}{1.5pt}
\setlength{\abovetopsep}{4pt}
\usepackage[margin=0.7in]{geometry}
\usepackage{nicefrac}
\usepackage{subcaption}
\newenvironment{psmallmatrix}
  {\left[\begin{matrix}}
  {\end{matrix}\right]}
  \usepackage{listings}


\begin{document}

\begin{center}
\begin{large}
BOSONS
\end{large}
\end{center}

\noindent Hamiltonian of $N$ $D$-dimensional bosons in harmonic oscillator potential with hard-sphere interaction ($\hbar = m = 1$):

\begin{align}
&\hat{H} = \sum_{i=1}^N \Big\{ -\frac{1}{2}\nabla_i^2 + \frac{1}{2}(\omega_1^2r_{i,1}^2+\cdots+\omega_{D}^2r_{i,D}^2) \Big\} + \sum_{i<j}^N V_{int}(\vec{r_i},\vec{r_j}),\\
&V_{int}(\vec{r_i},\vec{r_j}) =\left\{
                \begin{array}{ll}
                  \infty, \ r_{ij} \leq a \\
                  0, \ \ \ r_{ij} > a \\
                \end{array}
              \right.
\end{align}

\noindent Store the squares of the harmonic oscillator frequencies in a vector of size $D$ so we can write $\hat{H}$ as:
\begin{align}
&\hat{H} = \frac{1}{2} \sum_{i=1}^N \Big\{ -\nabla_i^2 + \vec{\omega^2}\cdot \vec{r_i^2} \Big\},\\
&\vec{\omega^2} =\vec{\omega} \ \% \ \vec{\omega} =  \langle \omega_1^2,\cdots, \omega_D^2 \rangle\\
&\vec{r_i^2} = \vec{r_i} \ \% \ \vec{r_i} =  \langle r_{i,1}^2,\cdots, r_{i,D}^2 \rangle\\
\end{align}

\noindent The interaction is taken care of by forcing the trial wavefunction to vanish if any two bosons become separated by a distance less than $a$. Trial wavefunction:

\begin{align}
&\Psi_T(\vec{R}) = \Big\{ \prod_{i=1}^N \exp(-\vec{\alpha}\cdot \vec{r_i^2} ) \Big\} \Big\{ \prod_{i<j}^N \exp \left( \ln \left( 1-\frac{a}{r_{ij}}\right) \right) \Big\}= \exp \left( p(\vec{\alpha},\vec{R}) + q(\vec{R}) \right)\\
&p(\vec{\alpha},\vec{R})= - \sum_{i=1}^N \vec{\alpha}\cdot\vec{r_i^2}\\
&q(\vec{R}) = \sum_{i<j}^N  \ln \left( 1-\frac{a}{r_{ij}}\right)
\end{align}

\noindent The local energy is defined as
\begin{align}
E_L(\vec{R}) = \frac{1}{\Psi_T}\hat{H}\Psi_T 
\end{align}

\noindent We need the following derivatives to obtain the analytical form of the local energy:
\begin{align*}
\frac{1}{\Psi_T}\nabla_i^2 \Psi_T &= \nabla_i^2 p + \nabla_i^2 q + (\nabla_i p + \nabla_i q)^2\\
&\nabla_i p = -2 \vec{\alpha} \ \% \ \vec{r_i}\\
&\nabla_i^2 p = -2 \vec{\alpha} \cdot \vec{\mathbbm{1}}\\
&\nabla_i q = \sum_{j \neq i}^N \frac{a}{r_{ij}^2(r_{ij}-a) }(\vec{r_i}-\vec{r_j})\\
&\nabla_i^2 q = \nabla_i \cdot \left( \sum_{j \neq i}^N \frac{a}{r_{ij}^2(r_{ij}-a)} (\vec{r_i}-\vec{r_j}) \right) \\
& \hspace*{7mm} = \sum_{d=1}^D \frac{\text{d}}{\text{d}r_{i,d}} \left( \sum_{j \neq i}^N \frac{a}{r_{ij}^2(r_{ij}-a) }(r_{i,d}-r_{j,d}) \right) \\
& \hspace*{7mm} = \sum_{j\neq i}^N \sum_{d=1}^D \frac{\text{d}}{\text{d}r_{i,d}} \left( \frac{a}{r_{ij}^2(r_{ij}-a) }(r_{i,d}-r_{j,d}) \right) \\
& \hspace*{7mm} = \sum_{j\neq i}^N \sum_{d=1}^D  \left( -\frac{a(r_{i,d}-r_{j,d})^2}{r_{ij}^4(r_{ij}-a)^2} (3r_{ij}-2a) + \frac{a}{r_{ij}^2(r_{ij}-a)}\right) \\
& \hspace*{7mm} = \sum_{j\neq i}^N \left( \frac{aD}{r_{ij}^2(r_{ij}-a)} -\sum_{d=1}^D 
\frac{a(r_{i,d}-r_{j,d})^2}{r_{ij}^4(r_{ij}-a)^2} (3r_{ij}-2a)  \right)\\
& \hspace*{7mm} = \sum_{j\neq i}^N \left( \frac{aD}{r_{ij}^2(r_{ij}-a)} -
\frac{a(3r_{ij}-2a)}{r_{ij}^4(r_{ij}-a)^2} \left(  \sum_{d=1}^D  (r_{i,d}-r_{j,d})^2 \right)  \right)\\
& \hspace*{7mm} = \sum_{j\neq i}^N \left( \frac{aD}{r_{ij}^2(r_{ij}-a)} -
\frac{a(3r_{ij}-2a)}{r_{ij}^2(r_{ij}-a)^2}  \right)\\
& \hspace*{7mm} = \sum_{j\neq i}^N \frac{a}{r_{ij}^2(r_{ij}-a)} \left( D - \frac{3r_{ij}-2a}{r_{ij}-a}   \right)\\
& \hspace*{7mm} = \sum_{j\neq i}^N \frac{a}{r_{ij}^2(r_{ij}-a)} \left( \frac{Dr_{ij}-Da -3r_{ij}+2a}{r_{ij}-a}   \right)\\
& \hspace*{7mm} = \sum_{j\neq i}^N \frac{a}{r_{ij}^2(r_{ij}-a)} \left( \frac{(D-3)r_{ij}+(2-D)a}{r_{ij}-a}   \right)\\
& \hspace*{7mm} = \sum_{j\neq i}^N \frac{a}{r_{ij}^2(r_{ij}-a)^2} \left( (D-3)r_{ij}+(2-D)a   \right)\\
& (\nabla_i p + \nabla_i q)^2 = \left( -2 \vec{\alpha} \ \% \ \vec{r_i} + \sum_{j\neq i}^N \frac{a(\vec{r_i}-\vec{r_j})}{r_{ij}^2(r_{ij}-a)} \right)^2\\
& \hspace*{23mm} = 4 \vec{\alpha^2}\cdot\vec{r_i^2} - 4a \sum_{j\neq i}^N \frac{(\vec{\alpha} \ \% \ \vec{r_i}) \cdot (\vec{r_i}-\vec{r_j})}{r_{ij}^2(r_{ij}-a)} + \left( \sum_{j\neq i}^N \frac{a(\vec{r_i}-\vec{r_j})}{r_{ij}^2(r_{ij}-a)} \right) \cdot \left( \sum_{j\neq i}^N \frac{a(\vec{r_i}-\vec{r_j})}{r_{ij}^2(r_{ij}-a)} \right)
\end{align*}



Putting all this together...
\begin{align*}
E_L(\vec{R}) &= \frac{1}{\Psi_T} 
\left[ \sum_{i=1}^N \Big\{ -\frac{1}{2}\nabla_i^2 + \frac{1}{2}(\omega_1^2r_{i,1}^2+\cdots+\omega_{D}^2r_{i,D}^2) \Big\} \right] \Psi_T \\
&= \frac{1}{2}  \sum_{i=1}^N \Big\{ -\frac{1}{\Psi_T} \nabla_i^2 \Psi_T + \vec{\omega^2} \cdot \vec{r_i^2}   \Big\}\\
&= \frac{1}{2} \sum_{i=1}^N \Big \{ - \nabla_i^2 p - \nabla_i^2 q - (\nabla_i p + \nabla_i q)^2 + \vec{\omega^2} \cdot \vec{r_i^2}   \Big\}\\
&= \frac{1}{2} \sum_{i=1}^N \Big \{ 2\vec{\alpha}\cdot \vec{\mathbbm{1}} - \sum_{j\neq i}^N \frac{a}{r_{ij}^2(r_{ij}-a)^2} \left( (D-3)r_{ij}+(2-D)a   \right) - \\
& \hspace*{5mm} \left[ 4 \vec{\alpha^2}\cdot\vec{r_i^2} - 4a \sum_{j\neq i}^N \frac{(\vec{\alpha} \ \% \ \vec{r_i}) \cdot (\vec{r_i}-\vec{r_j})}{r_{ij}^2(r_{ij}-a)} + \left( \sum_{j\neq i}^N \frac{a(\vec{r_i}-\vec{r_j})}{r_{ij}^2(r_{ij}-a)} \right) \cdot \left( \sum_{j\neq i}^N \frac{a(\vec{r_i}-\vec{r_j})}{r_{ij}^2(r_{ij}-a)} \right) \right] + \vec{\omega^2} \cdot \vec{r_i^2}   \Big\}\\
& = N\vec{\alpha}\cdot\vec{\mathbbm{1}} - \frac{1}{2}\sum_{i=1}^N \sum_{j\neq i}^N \frac{a((D-3)r_{ij}+(2-D)a)}{r_{ij}^2(r_{ij}-a)^2} - 2\sum_{i=1}^N \vec{\alpha^2}\cdot \vec{r_i^2} + 2a \sum_{i=1}^N \sum_{j \neq i}^N \frac{(\vec{\alpha} \ \% \ \vec{r_i})(\vec{r_i}-\vec{r_j})}{r_{ij}^2(r_{ij}-a)}\\
& \hspace*{5mm} -\frac{1}{2} \sum_{i=1}^N \left( \sum_{j\neq i}^N \frac{a(\vec{r_i}-\vec{r_j})}{r_{ij}^2(r_{ij}-a)} \right) \cdot \left( \sum_{j\neq i}^N \frac{a(\vec{r_i}-\vec{r_j})}{r_{ij}^2(r_{ij}-a)} \right) + \frac{1}{2}\sum_{i=1}^N \vec{\omega^2}\cdot \vec{r_i^2}\\
& = N\vec{\alpha}\cdot\vec{\mathbbm{1}} - 2\sum_{i=1}^N \vec{\alpha^2}\cdot \vec{r_i^2} 
+ \sum_{i<j}^N \left[ \frac{a((3-D)r_{ij}+(D-2)a)}{r_{ij}^2(r_{ij}-a)^2} + \frac{4a(\vec{\alpha} \ \% \ \vec{r_i})(\vec{r_i}-\vec{r_j})}{r_{ij}^2(r_{ij}-a)} \right]\\
& \hspace*{5mm} -\frac{1}{2} \sum_{i=1}^N \left( \sum_{j\neq i}^N \frac{a(\vec{r_i}-\vec{r_j})}{r_{ij}^2(r_{ij}-a)} \right) \cdot \left( \sum_{j\neq i}^N \frac{a(\vec{r_i}-\vec{r_j})}{r_{ij}^2(r_{ij}-a)} \right) + \frac{1}{2}\sum_{i=1}^N \vec{\omega^2}\cdot \vec{r_i^2}\\
& = N\vec{\alpha}\cdot\vec{\mathbbm{1}} - 2\sum_{i=1}^N \vec{\alpha^2}\cdot \vec{r_i^2} 
+ \sum_{i<j}^N \frac{a}{r_{ij}^2(r_{ij}-a)} \left[ 
\frac{(3-D)r_{ij}+(D-2)a}{r_{ij}-a} + 4(\vec{\alpha} \ \% \ \vec{r_i})\cdot (\vec{r_i}-\vec{r_j})
\right]\\
& \hspace*{5mm} -\frac{1}{2} \sum_{i=1}^N \left( \sum_{j\neq i}^N \frac{a(\vec{r_i}-\vec{r_j})}{r_{ij}^2(r_{ij}-a)} \right) \cdot \left( \sum_{j\neq i}^N \frac{a(\vec{r_i}-\vec{r_j})}{r_{ij}^2(r_{ij}-a)} \right) + \frac{1}{2}\sum_{i=1}^N \vec{\omega^2}\cdot \vec{r_i^2}\\
\end{align*}

\noindent For importance sampling, we need the gradient of $\Psi_T$ with respect to the variational parameters $\alpha_d$, $d=1,...,D$. 
\begin{align*}
\frac{\partial}{\partial \alpha_d} \Psi_T = \exp(p+q) \frac{\partial}{\partial \alpha_d} p = - \Psi_T \sum_{i=1}^N r_{i,d}^2
\end{align*}

\noindent For gradient descent, we need an expression for the derivative of $E_L$ with respect to the variational parameters $\vec{\alpha}$. We will only consider contributions from $b$ particles in batch $B_k$, where $k$ is chosen randomly. The variational parameters are only present in a few terms in the above expression for $E_L$, so we shall ignore the rest. 

\begin{align*}
\frac{\partial E_L}{\partial \alpha_d} 
&= \frac{\partial}{\partial \alpha_d} \left( 
b\vec{\alpha}\cdot \vec{\mathbbm{1}} - 2 \sum_{i \in B_k} \vec{\alpha^2}\cdot \vec{r_i^2} 
+ \sum_{i\in B_k} \frac{1}{2} \sum_{j\neq i}^N \frac{4a}{r_{ij}^2(r_{ij}-a)} (\vec{\alpha} \  \% \ \vec{r_i}) \cdot (\vec{r_i}-\vec{r_j})
 \right)\\
 &= b -4 \sum_{i \in B_k} \alpha_d r_{i,d}^2 +2 \sum_{i \in B_k} \sum_{j \neq i}^N \frac{a r_{i,d}(r_{i,d}-r_{j,d})}{r_{ij}^2(r_{ij}-a)}
\end{align*}



The quantum force on the $i$th particle is defined as 
\begin{align}
\vec{F}_i(\vec{R}) = 2 \frac{1}{\Psi_T} \nabla_i \Psi_T.
\end{align}

Using the derivatives we have already calculated, we have
\begin{align*}
\vec{F}_i(\vec{R}) &= 2 ( \nabla_i p + \nabla_i q)\\
&= -4\vec{\alpha} \ \% \ \vec{r_i} + 2 \sum_{j\neq i}^N \frac{a(\vec{r_i}-\vec{r_j})}{r_{ij}^2(r_{ij}-a)}  \\
\end{align*}

\noindent The Langevin and Fokker-Planck equations give a new position $y$ from the old position $x$:
\begin{align}
y = x + d\Delta tF(x) + \xi \sqrt{\Delta t},
\end{align}
where $d=0.5$ is the diffusion constant and $\Delta t \in [0.001,0.01]$ is a chosen time step. \\

The transition probability is given by the Green's function
\begin{align}
G(y,x)=\frac{1}{(4\pi d \Delta t)^{3N/2}} \exp \left( -\frac{(y-x-d\Delta t F(x))^2}{4 d \Delta t} \right),
\end{align}
so that the Metropolis-Hastings acceptance ratio is
\begin{align}
A(y,x) = \min \{ 1, P(y,x) \},
\end{align}
where 
\begin{align*}
P(y,x) &= \frac{G(x,y) | \Psi_T(y) | ^2}{G(y,x) | \Psi_T(x) | ^2}\\
&= \exp \left( -\frac{(x-y-d\Delta t F(y))^2}{4 d \Delta t} \right) \exp \left( \frac{(y-x-d\Delta t F(x))^2}{4 d \Delta t} \right) \frac{| \Psi_T(y) | ^2}{ | \Psi_T(x) | ^2}\\
&= \exp \left( -\frac{(x-y)^2-2(x-y)d\Delta t F(y) + d^2 \Delta t^2 F(y)^2}{4d\Delta t} \right)\\
& \ \ \  \times \exp \left( \frac{(y-x)^2-2(y-x)d\Delta t F(x) + d^2 \Delta t^2 F(x)^2}{4d\Delta t} \right) \frac{| \Psi_T(y) | ^2}{ | \Psi_T(x) | ^2}\\
&= \exp \left( \frac{2(x-y)d\Delta t (F(y)+F(x)) + d^2 \Delta t^2 (F(x)^2-F(y)^2)}{4d\Delta t} \right) \frac{| \Psi_T(y) | ^2}{ | \Psi_T(x) | ^2}\\
&= \exp \left( \frac{2(x-y) (F(y)+F(x)) + d \Delta t (F(x)^2-F(y)^2)}{4} \right) \frac{| \Psi_T(y) | ^2}{ | \Psi_T(x) | ^2}\\
&= \exp \left( \frac{1}{2}(x-y) (F(y)+F(x)) + \frac{1}{4} d \Delta t (F(x)^2-F(y)^2) \right) \frac{| \Psi_T(y) | ^2}{ | \Psi_T(x) | ^2}\\
\end{align*}





\end{document}
