\documentclass[12pt]{article}
\pdfoutput=1
\usepackage{bm}% bold math
\usepackage{graphicx}
\graphicspath{}
%\usepackage{tabularx}
\usepackage{amsfonts}
\usepackage{amsmath}
\usepackage{amssymb}
\usepackage{amsbsy}
\usepackage{bbm}
\usepackage{hyperref}
\usepackage{mathdots}
\usepackage{booktabs}
\setlength{\heavyrulewidth}{1.5pt}
\setlength{\abovetopsep}{4pt}
\usepackage[margin=0.7in]{geometry}
\usepackage{nicefrac}
\usepackage{subcaption}
\newenvironment{psmallmatrix}
  {\left[\begin{matrix}}
  {\end{matrix}\right]}
  \usepackage{listings}


\begin{document}

\begin{center}
\begin{large}
BOSONS
\end{large}
\end{center}

\noindent Hamiltonian of $N$ $D$-dimensional bosons in harmonic oscillator potential with hard-sphere interaction ($\hbar = m = 1$):

\begin{align}
&\hat{H} = \sum_{i=1}^N \Big\{ -\frac{1}{2}\nabla_i^2 + \frac{1}{2}(\omega_1^2r_{i,1}^2+\cdots+\omega_{D}^2r_{i,D}^2) \Big\} + \sum_{i<j}^N V_{int}(\vec{r_i},\vec{r_j}),\\
&V_{int}(\vec{r_i},\vec{r_j}) =\left\{
                \begin{array}{ll}
                  \infty, \ r_{ij} \leq a \\
                  0, \ \ \ r_{ij} > a \\
                \end{array}
              \right.
\end{align}

\noindent Store the squares of the harmonic oscillator frequencies in a vector of size $D$ so we can write $\hat{H}$ as:
\begin{align}
&\hat{H} = \frac{1}{2} \sum_{i=1}^N \Big\{ -\nabla_i^2 + \vec{\omega^2}\cdot \vec{r_i^2} \Big\},\\
&\vec{\omega^2} =\vec{\omega} \ \% \ \vec{\omega} =  \langle \omega_1^2,\cdots, \omega_D^2 \rangle\\
&\vec{r_i^2} = \vec{r_i} \ \% \ \vec{r_i} =  \langle r_{i,1}^2,\cdots, r_{i,D}^2 \rangle\\
\end{align}

\noindent The interaction is taken care of by forcing the trial wavefunction to vanish if any two bosons become separated by a distance less than $a$. Trial wavefunction:

\begin{align}
&\Psi_T(\vec{R}) = \Big\{ \prod_{i=1}^N \exp(-\vec{\alpha}\cdot \vec{r_i^2} ) \Big\} \Big\{ \prod_{i<j}^N \exp \left( \ln \left( 1-\frac{a}{r_{ij}}\right) \right) \Big\}= \exp \left( p(\vec{\alpha},\vec{R}) + q(\vec{R}) \right)\\
&p(\vec{\alpha},\vec{R})= - \sum_{i=1}^N \vec{\alpha}\cdot\vec{r_i^2}\\
&q(\vec{R}) = \sum_{i<j}^N  \ln \left( 1-\frac{a}{r_{ij}}\right)
\end{align}

\noindent The local energy is defined as
\begin{align}
E_L(\vec{R}) = \frac{1}{\Psi_T}\hat{H}\Psi_T 
\end{align}

\noindent We need the following derivatives to obtain the analytical form of the local energy:
\begin{align*}
\frac{1}{\Psi_T}\nabla_i^2 \Psi_T &= \nabla_i^2 p + \nabla_i^2 q + (\nabla_i p + \nabla_i q)^2\\
&\nabla_i p = -2 \vec{\alpha} \ \% \ \vec{r_i}\\
&\nabla_i^2 p = -2 \vec{\alpha} \cdot \vec{\mathbbm{1}}\\
&\nabla_i q = \sum_{j \neq i}^N \frac{a}{r_{ij}^2(r_{ij}-a) }(\vec{r_i}-\vec{r_j})\\
&\nabla_i^2 q = \nabla_i \cdot \left( \sum_{j \neq i}^N \frac{a}{r_{ij}^2(r_{ij}-a)} (\vec{r_i}-\vec{r_j}) \right) \\
& \hspace*{7mm} = \sum_{d=1}^D \frac{\text{d}}{\text{d}r_{i,d}} \left( \sum_{j \neq i}^N \frac{a}{r_{ij}^2(r_{ij}-a) }(r_{i,d}-r_{j,d}) \right) \\
& \hspace*{7mm} = \sum_{j\neq i}^N \sum_{d=1}^D \frac{\text{d}}{\text{d}r_{i,d}} \left( \frac{a}{r_{ij}^2(r_{ij}-a) }(r_{i,d}-r_{j,d}) \right) \\
& \hspace*{7mm} = \sum_{j\neq i}^N \sum_{d=1}^D  \left( -\frac{a(r_{i,d}-r_{j,d})^2}{r_{ij}^4(r_{ij}-a)^2} (3r_{ij}-2a) + \frac{a}{r_{ij}^2(r_{ij}-a)}\right) \\
\end{align*}


\end{document}
